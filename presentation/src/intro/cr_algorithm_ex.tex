\pgfplotsset{
  cycle list name=exotic,
  axis y line=left,
  axis x line=bottom,
}
\pgfplotsset{%
  colormap={rg}{color(0)=(lime!90!black); color(1)=(red!60!white)},
}

\pgfplotsset{likelihood plot/.style= {
    height=4cm,
    width=0.9\columnwidth,
    samples=40,
    domain=0:1,
    ymin=0,
    xmin=0,
    enlarge y limits=upper,
    every axis plot post/.append style={very thick},
    no markers,
    y tick label style={/pgf/number format/fixed,
      /pgf/number format/1000 sep = \thinspace % Optional if you want to replace comma as the 1000 separator
    },
    z tick label style={/pgf/number format/fixed,
      /pgf/number format/1000 sep = \thinspace % Optional if you want to replace comma as the 1000 separator
    },
  }}

\pgfplotsset{comb/.style= {dashed, draw=gray!40,
    every axis plot post/.append style=
    {mark=*,mark options={fill=lime!90!black,draw=gray!60, solid}} } }

\begin{frame}
  \frametitle{Candidate Ranking}
  \framesubtitle{Example}
  \monocolumn{
    \resizebox{\columnwidth}{!}{
      \begin{tikzpicture}[]
        \setlength{\tabcolsep}{0pt}%
        \renewcommand{\arraystretch}{1.2}%

        \tikzstyle{line} = [draw,rounded corners, -latex',ultra thick]
        \tikzstyle{box} = [inner sep=0pt,outer sep=0pt, rounded corners,align=center]
        \tikzstyle{inner box} = [box, clip]
        \tikzstyle{outer box} = [box, thick, draw]
        \tikzstyle{some space} = [inner sep=0.5em]

        \node[inner box] (a){%
          \begin{tabular}[m]{C{1cm}|*{3}{C{1cm}}|C{1cm}C{0cm}}
            \multirow{2}{*}{$i$} & \multicolumn{3}{c|}{$A_{ij}$} & \multirow{2}{*}{$e_i$}                                          \\\cline{2-4}
                                 & $c_1$                    & $c_2$             & $c_3$             &                       \\\hline%
            $1$     & \cellcolor{clrc}1        & \cellcolor{clrc}1 & 0                 & \cellcolor{clre}1  &\\
            $2$     & \cellcolor{clrc}1        & 0                 & \cellcolor{clrc}1 & \cellcolor{clre}1  &\\
            $3$     & \cellcolor{clrc}1        & 0                 & 0                 & \cellcolor{clrd}0  &
          \end{tabular}};
        \node[outer box,thin,fit=(a)] {};

        \node[inner box, below=of a,some space] (D){%
          $D = \Big\{\set{c_1}, \set{c_2,c_3}\Big\}$
        };

        \node[outer box,fit=(a) (D),some space] (a){};
        \onslide<+(1)->
        {
          \node[inner box, right= of a.east,some space,align=right]  (b){
            $\posterior[\set{c_1}] = \overbrace{ \frac{1}{1000} \cdot \frac{999}{1000} \cdot \frac{999}{1000} }^{\prior} \times \overbrace{\vphantom{\frac{1}{1}} \underbrace{(1-g_1)}_{t_1} \cdot \underbrace{(1-g_1)}_{t_2} \cdot \underbrace{g_1}_{t_3} }^{\likelihood}$
          };
          \node[outer box,fit=(b)] {};

          \draw[line] (a) -> (b);

          \node[outer box,below right= 0.7cm of b.south,some space] (c){
            \begin{tikzpicture}
              \begin{axis} [
                likelihood plot,
                colormap name=rg,
                xlabel={\large$g_1$},
                xtick={0,0.33,0.66,1},
                width=0.45\columnwidth
                ]
                \addplot+[mesh] {(1-x)*(1-x)*(x)};

                \addplot+[ycomb,comb] plot coordinates {(0.333,0.149)};
                \addplot+[xcomb,comb] plot coordinates {(0.333,0.149)};


              \end{axis}
            \end{tikzpicture}
          };
          \draw[line] (b) -> (c);
        }
        \onslide<+(1)->
{
          \node[outer box,below right= of a.south east,some space,anchor=west] (b2){$\posterior[\set{c_2, c_3}] = \cdots$};
          \draw[line] (a) edge[out=0,in=180] (b2);

          \node[outer box,below= 5cm of b.south,some space] (d){
            $\posterior[\set{c_1}] = 0.993$ \\
            $\posterior[\set{c_2,c_3}] = 0.007$
          };
          \draw[line] (b2) ->node[pos=0.5,outer box,some space,fill=white]{$\cdots$} (d);
          \draw[line] (c) -> (d);

        }
      \end{tikzpicture}
    }
  }


  \note{
    \begin{itemize}
      \compresslist
    \item Com este exemplo acabo então a intro da minha apresentação e
      vou  passar a descrever as minhas contribuições.
    \end{itemize}
  }
\end{frame}
