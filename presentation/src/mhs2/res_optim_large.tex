\renewcommand{\scaleig}[2][]{\includegraphics[scale=1,#1]{#2}}
\begin{frame}
  \frametitle{Optimizations -- Benchmark}
  \framesubtitle{$M,N = 10000$}

  \monocolumn{
    \begin{tikzbenchmarkfig}{}
      \node(main0){};
      \node(main){};
      \begin{scope}[local bounding box=scope1]

        \begin{pgfonlayer}{background}
          \alt<+(1-)>{
            \node[main plot,below=1cm of main.south, anchor=north] (main){%
              \scaleig[trim=1.2cm 1.15cm 0.53cm 2.1cm, clip, page=7]%
              {figures/mhs2/optim_large.pdf}%
            };
            \node[left=0cm of main.south west,anchor=south east](axisy){
              \scaleig[trim=0.7cm 1.15cm 9.07cm 2cm, clip,page=7]
              {figures/mhs2/optim_large.pdf}};
            \node[left=0.2cm of scope1.west,anchor=east]{
              \rotatebox{90}{Relative Improvement (log)}};
          }{
            \node[main plot,below=1cm of main.south, anchor=north] (main){%
              \scaleig[trim=1.2cm 1.15cm 0.53cm 2.1cm, clip, page=6]%
              {figures/mhs2/optim_large.pdf}%
            };
            \node[left=0cm of main.south west,anchor=south east](axisy){
              \scaleig[trim=0.7cm 1.15cm 9.07cm 2cm, clip,page=6]
              {figures/mhs2/optim_large.pdf}};
            \node[left=0.2cm of scope1.west,anchor=east]{
              \rotatebox{90}{Throughput (log)}};

          }

          \drawaxes{main}
          \node[below=0cm of main.south](axisx){
            \scaleig[trim=1.2cm 0.7cm 0.53cm 5.25cm, clip, page=6]
            {figures/mhs2/optim_large.pdf}};
        \end{pgfonlayer}
      \end{scope}

      \node[below=0.1cm of axisx.south]{$R$};


      \node[above= -0.4cm of main0.north, anchor=south,align=center] (capt){
        \scaleig[trim=0.9cm 4.5cm 0.5cm 1cm, clip, page=6]
        {figures/mhs2/optim_large.pdf}};
    \end{tikzbenchmarkfig}
  }
  \note{
    \begin{itemize}
      \compresslist
    \item Cutoff: 30 secs
    \item Se normalizarmos... melhoria de:
      \begin{itemize}
      \item avg: 315 $\times$
      \end{itemize}
    \item A melhoria é maior quanto maior o problema
    \end{itemize}
  }
\end{frame}
