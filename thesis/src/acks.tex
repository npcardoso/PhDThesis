\makeatletter
\renewcommand{\@chapapp}{}% Not necessary...
\newenvironment{chapquote}[2][2em]
{\setlength{\@tempdima}{#1}%
  \def\chapquote@author{#2}%
  \parshape 1 \@tempdima \dimexpr\textwidth-2\@tempdima\relax%
  \itshape}
{\par\normalfont\hfill--\ \chapquote@author\hspace*{\@tempdima}\par\bigskip}
\makeatother
\setcounter{footnote}{0}
\renewcommand{\BrainFuckChapter}{
  {-}{-}{-}{-}{[}{-}{-}{-}{-}{>}{+}{<}{]}{>}{+}{+}{.}{[}{-}{-}{-}{>}{+}{<}{]}{>}{-}{-}{-}{-}{.}{+}{+}{+}{+}{+}{+}{+}{+}{+}{+}{+}{.}{+}{+}{+}{[}{-}{>}{+}{+}{+}{<}{]}{>}{+}{+}{.}{+}{+}{+}{.}{+}{.}{-}{-}{.}{+}{+}{+}
  {+}{+}{+}{.}{+}{+}{+}{+}{.}{-}{-}{-}{-}{-}{-}{-}{-}{.}{+}{+}{+}{+}{+}{+}{+}{+}{+}{.}{+}{+}{+}{+}{+}{+}{.}{-}{-}{-}{-}{-}{.}{+}{+}{+}{+}{.}{[}{-}{-}{-}{>}{+}{<}{]}{-}{-}{-}{-}{+}{+}{+}{+}{+}{+}{+}{+}{+}{>}{>}{>}
}
\chapter*{Agradecimentos}
E cá está... o resultado de quase 5 anos da minha vida.

Embora possa parecer tudo muito mecânico e trivial, a realidade é que
a formalidade da escrita oculta toda a complexidade da experiência
vivida.
%
Houve momentos de imensa felicidade e momentos de enorme desespero.
%
Gosto de fazer a comparação do processo com o Paradoxo de Zeno:

\begin{chapquote}{Wikipedia\footnote{À qual gostaria também de expressar o meu sincero agradecimento.}}
  ``Suppose Homer wishes to walk to the end of a path. Before he can
  get there, he must get halfway there. Before he can get halfway
  there, he must get a quarter of the way there. Before traveling a
  quarter, he must travel one-eighth; before an eighth, one-sixteenth;
  and so on.  (...)  This description requires one to complete an
  infinite number of tasks, which Zeno maintains as an
  impossibility.''
\end{chapquote}

Estabelecendo o paralelismo, inicialmente (\ie, imediatamente após a
depressão de se perceber que não se capta nada do assunto), com o
entusiasmo e energia de quem começa, a tarefa era agradável e grandes
progressos foram feitos.
%
No entanto, à medida que o tempo foi passando e a energia se foi
esgotando, a tarefa tornou-se cada vez mais penosa e, apesar de, na
prática, o objetivo final estar cada vez mais próximo, a minha
perceção era de que eu não iria conseguir.
%
O fim de cada tarefa dava origem ao início de outra, a qual era
desempenhada com menor eficiência que a tarefa antecedente.
%
Apesar de estar extremamente grato pela experiência, reconheço que foi
bastante mais difícil do que eu esperava.
%
Tenho a certeza de que a concretização desta tarefa só foi possível
por ter tido ao meu lado as pessoas a quem passo a agradecer.
%


Gostaria de começar por agradecer à minha melhor amiga e amor da minha
vida, Lígia ``Maria'' Massena, por ter estado ao meu lado e me ter
apoiado em todos os momentos deste desafio.
%



Ao meu amigo e orientador, Rui ``Grande Líder Kim Jong-Rui'' Abreu,
por acompanhar de perto todo este percurso e ter a paciência
necessária para lidar comigo.
%
Fico extremamente grato por ter tido um ``chefe'' que me tenha dado
não só a liberdade mas também o incentivo para explorar todos os meus
hobbies.



Aos meus ídolos, o meu Avô Licínio, o meu amigo de longa data Nuno
Cunha, e o meu grande amigo Vítor Almeida.
%
As suas experiências de vida foram e continuam a ser uma fonte de
inspiração para mim.
%
Em particular, o meu Avô inspira-me pela sua determinação e
tenacidade, o Nuno pelo seu sentido de justiça, integridade e
conhecimento enciclopédico, e o Vítor pela sua generosidade e incrível
engenho.

À minha sogra, Iolanda ``Sogra'' Macena, que é mais do que uma mãe
para mim e com quem partilho uma profunda amizade, pelo seu apoio
incondicional.

Ao meu sogro, Horácio ``Macho'' Massena, que apesar do início de
relação ``tenso'', é uma pessoa que atualmente me trata como filho e
tem um orgulho genuíno dos meus feitos.

À minha mãe e à minha Avó Olivinha, pelo apoio dado durante estes anos.

À minha ``avógra'', Olga ``Companheira'' Chicket, pelo seu papel
instrumental numa fase de grande adversidade.

Aos padrinhos, Mário ``Padrinho'' Tavares e Conceição ``Madrinha''
Tavares, pelo carinho e por marcarem presença em momentos
significativos, ainda que pudessem ser de pouco interesse para eles
(\eg, a defesa da minha tese).


Ao meu ``bro'', André ``Beat'' Silva, que esteve presente na grande
maioria dos momentos determinantes do meu doutoramento.
%
Foi um elemento estruturante neste processo e a quem devo uma grande
parte dos meus sucessos.
%
Partilhámos uma enorme diversidade de vivências, desde momentos de
alegria extrema (\eg, ``quod erat demonstrandum'', ``colar o pistão'',
\etc), a momentos de puro pânico (\eg, o episódio do gringo possuído,
o evento pós-telepizza, \etc).
%
Num aspeto mais mundano, e entre muitas outras coisas, o Beat
inspirou-me a construir uma bateria e a aprender a tocá-la, coisa
que atualmente contribui bastante para a minha felicidade.

Ao meu companheiro de final de corrida, Lúcio ``Primaço'' Passos, pelo
seu apoio, motivação e inesgotável boa disposição. Foi bom ter alguém
igualmente miserável para poder perceber o meu ponto de vista.


Ao meu amigo de sempre, João Pedro ``Musty'' Dias, por todo o apoio
dado e por me tentar ensinar, ainda que com sucesso limitado, a
importância de relaxar: ``O Musty de amanhã pensa no assunto...''.

Ao José Luís ``Oh, Cala-te!'' Pereira e à Neuza ``Oh, Cala-te!'' Florim,
um casal com quem, de forma rápida e inesperada, criei uma forte
amizade.
%
Apesar da distância, estão lá no fundo do meu coração robótico.

As minhas discussões com o Musty e com o Zé e a Neuza, por estarem
sempre assentes em formas muito diferentes de ver o mundo\footnote{A
  minha visão: $0110011010010011100$. A deles: arcos-íris e unicórnios
  (caricaturado).} obrigaram-me a reconsiderar muitas ``verdades
absolutas''.
%
Em retrospetiva, sinto que o facto de atualmente considerar a
existência de um $0.5$ se deve em grande parte a eles.


Ao Márcio ``Márcinho'' Sá, companheiro de bricolage, por estar sempre
disponível para me ajudar, ao ponto de se matar (quase literalmente) a
trabalhar.

Ao Elói ``Braahh'' Barros e Nuno ``Té Looogooo...'' Gaspar pelo apoio,
boa disposição e prontidão para o que desse e viesse. Té looogooooo!!!


À Raquel Almeida, com quem vivi uma experiência que me ficará marcada
para o resto da vida, e no processo me ensinou a reconhecer o valor
dos amigos.

À Idalina ``Lina'' Silva, pelos seus valiosos conselhos e por me ter
facilitado imensamente a vida ao resolver-me prontamente problemas que
estavam para alem das suas obrigações (\eg, reservar/fazer check-in em
voos para conferências, enviar-me documentos para a FCT, \etc).


To Luuk Eliens, a guy with whom I immediately felt connected to and
had the pleasure to work with. I truly admire his determination when
pursuing a goal. I hope one day we can work together again.

Ao Vítor e Marly Pinto, pelo seu carinho e amizade e por me brindarem
consistentemente com alegria e boa disposição.

Ao Mauari Quintero, à Katherine Velosa e à ``Sammy'', uns amigos
improváveis, que me deram todo o seu carinho e apoio.

Ao Vasco Mota e à Isabel Pinto, pelo companheirismo e experiências
vividas em conjunto.

Ao Carlos ``Carlinhos'' Gouveia, que foi um gajo para quem eu
trabalhei uma tarde e, em troca, recebi uma enorme consideração e
reconhecimento que prevaleceram ao teste do tempo.

Às meninas do Subway, Maria Mendes, Sara Moreira, Marisa Soares, Ana
Albergaria, e ao Rúben Quintal, pela boa disposição, simpatia e,
sobretudo, pelas enumeras vezes que me alimentaram (com
``extra-pimentos'') durante estes anos.
